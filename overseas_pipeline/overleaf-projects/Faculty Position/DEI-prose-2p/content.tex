\noindent I share my experience regarding diversity, equity, and inclusion, and how I plan to continue pursuing them. 


% \begin{center}
% \textbf{\Large My Experience with Diversity, Equity, and Inclusion}\\[-6pt]
% % \rule{\textwidth}{1pt} % 横线长短与粗细可调整
% \end{center}

\begin{flushleft}
\textbf{\Large My Experience with Diversity, Equity, and Inclusion}\\[-6pt]
% \rule{\textwidth}{1pt} % 横线长短与粗细可调整
\end{flushleft}



\noindent \textbf{Supporting Female Students in Learning Computer Science.}
Computer science remains one of the least diverse fields in post-graduate education. For example, women represent only around 20\% of computer science PhD recipients in the United States\footnote{\url{https://ncses.nsf.gov/pubs/nsf21321/report/field-of-degree-women}}. This lack of representation highlights the ongoing need to create more inclusive and supportive environments for underrepresented groups in computing. There are impressions that \href{https://quillette.com/2018/06/19/why-women-dont-code/}{women do not code}. I was fortunate to study in interdisciplinary programs where the gender ratio was relatively balanced—fields such as information product design and software engineering. Through these experiences, I learned that students of all genders can excel in technical domains \textbf{when provided with equitable opportunities and supportive environments}. However, I also witnessed how informal networks and implicit biases can sometimes make women feel like outsiders, particularly in male-dominated spaces. Differently, my experience conducting Human–Computer Interaction (HCI) research has led me to realize that HCI, a research area within computer science, can serve as an accessible and inclusive entry point into computing. HCI attracts students from diverse disciplinary and personal backgrounds, many of whom might not have formal computer science training but are eager to contribute to technology design. 
Among the students I have mentored, six were women who came from varied fields such as design, psychology, and architecture. I have found great joy in mentoring these students—helping them build confidence in programming, research design, and interdisciplinary collaboration. One of my mentees, Zhiyao Shu, originally trained in architecture, decided to pursue computer science after discovering her passion for HCI research. I guided her through this transition by helping her strengthen her computational skills and by framing her interdisciplinary background as a unique strength. In our project on AI-assisted qualitative data analysis tool development, she served as the main developer, and her work on the final system received remarkable appreciation from users and colleagues alike. 

\noindent \textbf{Equity Experience as a Student from a Low-Resource Academic Environment.}
During my Ph.D. at the Singapore University of Technology and Design (SUTD), I became acutely aware of how institutional resources can shape students’ learning and research opportunities. As a young university, SUTD offered limited support for certain emerging fields such as Human–Computer Interaction (HCI), compared to more established domains like natural language processing. Through research visits to the NUS-HCI Lab at the National University of Singapore and to the University of Notre Dame in the United States, I experienced how different environments can foster different kinds of growth. At Notre Dame, under the mentorship of Prof. Toby Li, I observed how students benefited from strong institutional support for conference travel, cross-group collaboration, and community building. At the NUS-HCI Lab, I saw the value of frequent peer feedback sessions and collective learning. Later, during my postdoc at SMART center and Johns Hopkins University, I again witnessed how access to well-developed networks and mentoring structures can profoundly influence students’ research confidence and career development. These experiences gave me a clear understanding of how unequal access to academic resources—such as mentoring, funding, and professional exposure—can lead to unequal outcomes for students. They also deepened my belief that equitable opportunities and supportive environments enable every student to achieve what they might not have thought possible. As an educator and mentor, I strive to build such environments by offering structured guidance, peer feedback, and access to professional growth opportunities, regardless of a student’s background or institutional context. \\


% \noindent \textbf{Building A Group for Diverse Opinions.} During my postdoctoral research at the SMART Centre, I also practiced inclusive recruitment in my lab. When selecting interns, I intentionally built a team with diverse backgrounds and genders. The combination of a student with strong communication and design skills and another with deep programming expertise enriched our group’s collaboration and creativity. These experiences reaffirmed my belief that diversity---in gender, discipline, and perspective---drives innovation. \\




\vspace{-20pt}
\begin{flushleft}
\textbf{\Large My Plans for Diversity, Equity, and Inclusion}\\[-6pt]
% \rule{\textwidth}{1pt} % 横线长短与粗细可调整
\end{flushleft}


\noindent Looking ahead, I am committed to creating inclusive learning and research environments. I will continue to mentor students from underrepresented backgrounds, ensure equitable opportunities in my lab, and advocate for inclusive conference participation and research collaboration practices. I believe that diversity is not only about representation---it is about empowering every individual to contribute their unique perspectives to advance knowledge and make computing more humane. \\

\noindent \textbf{Recruiting Diverse Members.}
Benefiting from my previous experience in interdisciplinary research and the creativity gained from working across diverse domains, I plan to build a research group that embraces diversity in every aspect. I will recruit PhD students and visiting scholars from both psychology and computer science, and I am committed to supporting members of all genders and LGBTQ+ identities. I believe a research lab should not only be a place where I mentor students, but also a diverse and inclusive community where everyone can learn from one another’s unique perspectives and experiences. \\


\noindent \textbf{Supporting Equal Environment.} My lab will provide equal opportunities for all students, regardless of gender or identity. For example, I will ensure that female and LGBTQ+ students have equal access to opportunities to present their work, develop their skills, and receive resources and mentorship. My goal is to create an environment where everyone can thrive and contribute equally. \\

\noindent \textbf{Building Inclusive and Supportive Culture.}
My lab will care not only for students who already demonstrate academic excellence, but also for those who are still developing their research abilities. Every student deserves encouragement and equal treatment. This approach not only supports their growth as researchers but also fulfills a mentor’s responsibility to help them mature as learners and individuals.


