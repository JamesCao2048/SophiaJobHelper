
\noindent As an Asian woman who has studied and conducted research in China, Singapore, and the US, I deeply understand the obstacles faced by women and other minorities, especially those from different cultural backgrounds. I believe that everyone should have the opportunity to participate in and contribute to the fields of computer science and artificial intelligence, regardless of socioeconomic status, race, gender, country of birth, or any other factor. Below, I share my past experiences, as well as my ongoing and future efforts to promote diversity, equity, and inclusion.



\begin{flushleft}
\noindent \textbf{\Large Diversity Experience and Supporting Efforts}\\[-6pt]
\end{flushleft}

\noindent During my long educational journey, there were \textbf{three key moments when many women, including myself, faced barriers to continuing higher education}. \textbf{The first moment} occurred in middle school, when teachers frequently told us that female students perform well because success at that stage relies on memorization, while male students would later excel in subjects requiring logic and reasoning. This \textbf{implicit bias} discouraged many of my female classmates from pursuing further studies. \textbf{The second moment} came in high school, where girls were often steered toward subjects such as history or politics rather than physics, chemistry, or mathematics, which are courses that serve as essential foundations for higher education in computing and AI. This made the gap between genders in opportunities even wider. \textbf{The third moment} emerged during my undergraduate studies, when I witnessed many female peers losing confidence due to long-standing stereotypes suggesting that women are less capable of coding or tackling technically demanding problems.\footnote{\url{https://quillette.com/2018/06/19/why-women-dont-code/}} Consequently, many may choose not to pursue postgraduate studies. 

\noindent My personal experience mirrors a broader global pattern: computer science remains one of the least diverse fields in higher education. In the United States, for instance, women constitute only about 20\% of computer science PhD recipients.\footnote{\url{https://ncses.nsf.gov/pubs/nsf21321/report/field-of-degree-women}} Yet my own journey in computer science has shown that \textbf{students of all genders can thrive when provided with equitable opportunities and supportive environments}. These experiences reinforce my commitment to fostering inclusive academic spaces that empower underrepresented groups to succeed in computing.

\noindent In response, I am currently engaged in and will continue my supporting efforts, as a faculty member, by: \\[-25pt]
\begin{itemize}
    \item \textbf{Recruiting and mentoring diverse members.} When I was recruiting and mentoring interns, I intentionally practiced inclusive mentorship. Among the 10+ students I have mentored, six were women. For projects, I maintained gender balance within our research teams, encouraged open idea exchange, and designed collaborative tasks that fostered mutual learning. I have found great joy in mentoring a diverse team and was often inspired by the perspectives of people from different backgrounds. In the future, I will continue to recruit and mentor members from underrepresented groups, such as female and LGBTQ+ students, ensuring that every voice is valued. Through these efforts, I aim to create a research culture where diversity becomes a driver of creativity and innovation.
    \\[-20pt]
    \item \textbf{Extending diversity through research.} Beyond mentorship, my research direction, Human–AI Collaboration (HAI), further reinforces diversity by structurally attracting students from underrepresented backgrounds. HAI research integrates human-centered perspectives with computational methods, creating an inclusive pathway into computing for students from fields such as psychology and the social sciences. These disciplines tend to have more balanced gender representation compared to computer science, enabling broader participation. By transforming their unique disciplinary and cultural backgrounds into strengths, these students can gain confidence and make meaningful contributions to computing and AI research. In this way, my research not only advances technology but also builds bridges that expand who can participate in it. \\[-20pt]
\end{itemize}

\begin{flushleft}
\noindent \textbf{\Large Equity Experience and Supporting Efforts}\\[-6pt]
\end{flushleft}

\noindent While I was fortunate to have strong encouragement from my mother and uncle to pursue higher education, even my mother, the person who supported me the most, did not fully understand why I wanted to pursue a PhD. To her, a stable teaching position seemed more practical than a research career that offers less financial stability. This limited understanding also meant that I had \textbf{limited early access} to higher education information and opportunities. It was not until my master’s studies that I was first exposed to research projects and began to see a broader academic world. During my PhD, I came to realize that institutional inequality is not only about who gets admitted, but also about who receives the mentorship, funding, and professional visibility necessary to thrive. As a young university outside the United States, SUTD offered limited support for emerging fields such as Human Computer Interaction (HCI). Through research visits to National University of Singapore and University of Notre Dame, and later through my postdoctoral experiences at the SMART Centre and Johns Hopkins University, I experienced firsthand how different institutional environments foster different kinds of growth. I saw the value of frequent peer feedback and collective learning, as well as how well developed mentoring networks, travel support, and research communities strengthened students’ confidence and professional development. These experiences deepened my belief that \textbf{equitable opportunities, through mentoring, access, and institutional support, can empower students to achieve beyond what they imagine possible}. 

\noindent As a future faculty member, I aim to extend these principles by creating environments where all students, regardless of background, have access to the resources and guidance they need to succeed. Specifically: \\[-25pt]
\begin{itemize}
    \item \textbf{Providing formal learning opportunities.} Drawing on my cross-institutional experiences, I aim to offer access through opportunities, such as visiting student positions and cross-institutional collaborations, regardless of a student’s background or institutional context. These can help students with limited local resources access competitive opportunities in academia and beyond.
    \\[-20pt]
    \item \textbf{Providing informal learning opportunities.} I recognize that formal opportunities may not always be feasible. Therefore, I aim to actively engage in informal support, such as giving talks, organizing learning workshops, sharing transparent information, and responding to emails from low-resource students. These efforts make learning and mentorship more accessible to students around the world.
    \\[-20pt]
\end{itemize}

\begin{flushleft}
\noindent \textbf{\Large Inclusion Experience and Supporting Efforts}\\[-6pt]
\end{flushleft}

\noindent During my PhD journey, I was deeply supported by my advisor, who encouraged every member of our lab at SUTD regardless of nationality, language background, or prior expertise. When I first joined as a PhD student, my spoken English was limited, but he actively adjusted his communication style and provided continuous opportunities for me to improve. This inclusive mentorship profoundly shaped my understanding of what genuine academic support means, especially for early-career researchers. 
Beyond SUTD, during my research visits at NUS and Notre Dame and postdoctoral experiences at SMART and JHU, I was mentored by faculties who valued not only my research directions but also my learning potential. The encouragement I received throughout these formative years played a crucial role in my development as an independent researcher and continues to guide my own mentoring philosophy. Through shadowing faculty members at these institutions, I observed how inclusive mentorship practices can empower students from diverse academic and cultural backgrounds to build confidence and thrive in their research.



\noindent As a future faculty member, I will be committed to continuing my efforts to support students by: \\[-25pt]

\begin{itemize}
    \item \textbf{Fostering inclusivity by recognizing diverse strengths.} I believe every student possesses unique strengths. Some may excel in visible areas, such as communication or presentation, while others demonstrate excellence in less visible forms, such as perseverance, creativity, or technical depth. I will ensure that each team member can contribute by leveraging their individual strengths, fostering a sense of ownership, confidence, and belonging within the group.
    \\[-20pt]
    \item \textbf{Supporting students with limited resources or prior privileges.} I believe every student deserves equal treatment and the opportunity to voice their ideas. Therefore, my lab will not only support students who already demonstrate academic excellence but will also provide encouragement and opportunities for those who are still developing their research abilities. This approach supports their growth as researchers while fulfilling a mentor’s responsibility to guide them toward maturity as independent thinkers and contributors.
\end{itemize}