\documentclass[11pt, a4paper]{article}

% ============================================
% PACKAGES
% ============================================
\usepackage[utf8]{inputenc}
\usepackage[T1]{fontenc}
\usepackage{mathptmx}
\usepackage[top=0.65in, bottom=0.65in, left=0.75in, right=0.75in]{geometry}
\usepackage{titlesec}
\usepackage{xcolor}
\usepackage{hyperref}
\usepackage{enumitem}
\usepackage{fancyhdr}
\usepackage{lastpage}
\usepackage{parskip}
\usepackage{microtype}

% ============================================
% COLOR DEFINITIONS (Johns Hopkins Blue Theme)
% ============================================
\definecolor{accentcolor}{RGB}{0, 45, 114}
\definecolor{linkcolor}{RGB}{0, 114, 178}
\definecolor{textgray}{RGB}{60, 60, 60}
\definecolor{lightgray}{RGB}{128, 128, 128}

% ============================================
% HYPERREF SETUP
% ============================================
\hypersetup{
    colorlinks=true,
    linkcolor=linkcolor,
    urlcolor=linkcolor,
    citecolor=linkcolor,
    pdftitle={Diversity Statement - Jie Gao},
    pdfauthor={Jie Gao}
}

% ============================================
% HEADER AND FOOTER
% ============================================
\pagestyle{fancy}
\fancyhf{}
\renewcommand{\headrulewidth}{0pt}
\fancyhead[L]{\small\color{lightgray}Jie Gao}
\fancyhead[R]{\small\color{lightgray}Diversity Statement}
\fancyfoot[C]{\small\color{lightgray}\thepage/\pageref{LastPage}}

% ============================================
% SPACING
% ============================================
\setlength{\parskip}{0.4em}
\setlength{\parindent}{0em}

% ============================================
% SECTION FORMATTING
% ============================================
\titleformat{\section}
    {\normalsize\bfseries\color{accentcolor}}
    {}
    {0em}
    {}

\titlespacing*{\section}{0pt}{0.6em}{0.2em}

% ============================================
% DOCUMENT
% ============================================
\begin{document}

% ============================================
% HEADER
% ============================================
\thispagestyle{empty}
\begin{center}
    {\LARGE\bfseries\color{accentcolor}Diversity Statement}\\[0.4em]
    {\large Jie Gao}\\[0.2em]
    {\small\color{textgray}
    Malone Postdoctoral Fellow, Johns Hopkins University\\
    \href{mailto:jgao77@jh.edu}{jgao77@jh.edu} $\cdot$ \href{https://gaojie058.github.io}{gaojie058.github.io}
    }
\end{center}

\vspace{0.3em}

% ============================================
% INTRODUCTION
% ============================================
As an Asian woman who has studied and conducted research in China, Singapore, and the US, I deeply understand the obstacles faced by women and other minorities, especially those from different cultural backgrounds. I believe that everyone should have the opportunity to participate in and contribute to the fields of computer science and artificial intelligence, regardless of socioeconomic status, race, gender, country of birth, or any other factor.

% ============================================
% DIVERSITY
% ============================================
\section{Diversity}

As a faculty member, I aim to engage in and sustain efforts to support diversity by:

\begin{itemize}[leftmargin=1.5em, itemsep=0.3em, topsep=0.2em]
    \item \textbf{Recruiting and mentoring diverse members.} When I was recruiting and mentoring interns, I intentionally practiced inclusive mentorship. Among the 10+ students I have mentored, six were women. For projects, I maintained gender balance within our research teams, encouraged open idea exchange, and designed collaborative tasks that fostered mutual learning. I have found great joy in mentoring a diverse team and was often inspired by the perspectives of people from different backgrounds. In the future, I will continue to recruit and mentor members from underrepresented groups, such as female and LGBTQ+ students, ensuring that every voice is valued. Through these efforts, I aim to create a research culture where diversity becomes a driver of creativity and innovation.
    
    \item \textbf{Extending diversity through research.} Beyond mentorship, my research direction, Human--AI Collaboration (HAI), further reinforces diversity by structurally attracting students from underrepresented backgrounds. HAI research integrates human-centered perspectives with computational methods, creating an inclusive pathway into computing for students from fields such as psychology and the social sciences. These disciplines tend to have more balanced gender representation compared to computer science, enabling broader participation. By transforming their unique disciplinary and cultural backgrounds into strengths, these students can gain confidence and make meaningful contributions to computing and AI research. In this way, my research not only advances technology but also builds bridges that expand who can participate in it.
\end{itemize}

% ============================================
% EQUITY
% ============================================
\section{Equity}

As a faculty member, I aim to create environments where all students, regardless of background, have access to the resources and guidance they need to succeed. Specifically:

\begin{itemize}[leftmargin=1.5em, itemsep=0.3em, topsep=0.2em]
    \item \textbf{Providing formal learning opportunities.} Drawing on my cross-institutional experiences, I aim to offer access through opportunities, such as visiting student positions and cross-institutional collaborations, regardless of a student's background or institutional context. These can help students with limited local resources access competitive opportunities in academia and beyond.
    
    \item \textbf{Providing informal learning opportunities.} I understand that formal opportunities may not always be feasible. Therefore, I aim to actively engage in informal support, such as giving talks, organizing learning workshops, sharing transparent information, and responding to emails from low-resource students. These efforts make learning and mentorship more accessible to students around the world.
\end{itemize}

% ============================================
% INCLUSION
% ============================================
\section{Inclusion}

As a faculty member, I will be committed to support students by:

\begin{itemize}[leftmargin=1.5em, itemsep=0.3em, topsep=0.2em]
    \item \textbf{Fostering inclusivity by recognizing diverse strengths.} I believe every student possesses unique strengths. Some may excel in visible areas, such as communication or presentation, while others demonstrate excellence in less visible forms, such as perseverance, creativity, or technical depth. I will ensure that each team member can contribute by leveraging their individual strengths, fostering a sense of ownership, confidence, and belonging within the group.
    
    \item \textbf{Supporting students with limited resources or prior privileges.} I believe every student deserves equal treatment and the opportunity to voice their ideas. Therefore, my lab will not only support students who already demonstrate academic excellence but will also provide encouragement and opportunities for those who are still developing their research abilities. This approach supports their growth as researchers while fulfilling a mentor's responsibility to guide them toward maturity as independent thinkers and contributors.
\end{itemize}

\end{document}
